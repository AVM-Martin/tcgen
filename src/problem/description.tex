\section*{
    \centering
    Problem Name
}

\par Cillum nulla id in anim sunt id occaecat ut non anim proident dolor in reprehenderit in ut do cupidatat enim anim quis incididunt eiusmod eiusmod et excepteur commodo eiusmod in eu sit ut sint non duis dolor fugiat anim qui enim in culpa ex consectetur cillum est elit tempor anim ut commodo labore esse ad ea magna veniam ullamco minim velit et in ut laboris nulla esse aute anim mollit incididunt quis dolor magna veniam reprehenderit id ad in laboris non officia magna ullamco exercitation voluptate in ut est et adipisicing velit ut laborum officia consectetur dolore reprehenderit ad ut ut pariatur exercitation anim ut non ad consequat proident proident
\newline

\par Diberikan suatu array $A$ yang berisi $N$ elemen yang dinomori dari $1$ sampai $N$. Elemen ke-$i$ pada $A$ dinyatakan dengan $A_i$. Setiap $A_i$ terdiri dari $2$ bilangan bulat $X_i$ dan $Y_i$. Terdapat fungsi $F(a,b)$ yang mengembalikan hasil perkalian dari $a$ dan $b$. Sebagai contoh, $F(2,3)$ mengembalikan nilai $6$, sedangkan $F(7,3)$ mengembalikan nilai $21$. Terdapat juga array $B$ yang berisi elemen-elemen array $A$ yang telah diurutkan. $A_i$ memiliki posisi lebih awal dari $A_j$ pada $B$ apabila salah satu hal berikut terpenuhi:

\begin{itemize}
    \item $F(X_i,Y_i) > F(X_j, Y_j)$.
    \item $F(X_i,Y_i) = F(X_j, Y_j)$ dan $i < j$.
\end{itemize}

\par Terdapat juga fungsi $G(i)$ yang mengembalikan posisi $A_i$ pada $B$. Hitung nilai dari $G(1), G(2), \dots , G(N)$.




\subsection*{Format Input}

\par\noindent Baris pertama berisi sebuah bilangan bulat $N$. $N$ baris berikutnya masing-masing berisi dua buah bilangan bulat yang menyatakan nilai $X_i$ dan $Y_i$.




\subsection*{Format Output}

\par\noindent Keluarkan $N$ baris. Baris ke-$i$ menyatakan nilai $G(i)$.




\subsection*{Constraint}

\begin{itemize}
    \item $1 \leq N \leq 5.000$
    \item $0 \leq X_i, Y_i \leq 10^{5.000}$
    \item Terdapat suatu bilangan $C$ sehingga untuk setiap $i$, $X_i + Y_i = C$
\end{itemize}




\subsection*{Sample Input}

\begin{lstlisting}
5
1 5
2 4
3 3
4 2
5 1
\end{lstlisting}




\subsection*{Sample Output}

\begin{lstlisting}
4
2
1
3
5
\end{lstlisting}




\subsection*{Explanation}

\par\noindent Array $B$ pada contoh masukan diatas:
\begin{lstlisting}
3 3
2 4
4 2
1 5
5 1
\end{lstlisting}
